\documentclass[11pt]{article}
\usepackage{geometry}                % See geometry.pdf to learn the layout options. There are lots.
\geometry{letterpaper}                   % ... or a4paper or a5paper or ... 
\usepackage[parfill]{parskip}    % Activate to begin paragraphs with an empty line rather than an indent
\usepackage{graphicx}
\usepackage{amssymb}
\usepackage{epstopdf}
\usepackage{color}
\DeclareGraphicsRule{.tif}{png}{.png}{`convert #1 `dirname #1`/`basename #1 .tif`.png}

\newcommand{\forceindent}{\leavevmode{\parindent=2em\indent}}

\newcounter{response}
\newenvironment{response}[1][]{\color{blue}\refstepcounter{response}\noindent{{{\bf{Response \arabic{response}.}}}} \normalfont #1}

\renewcommand{\theresponse}{{\bf Response \arabic{response}}}

\usepackage{url}

\title{Response to reviewer comments to the manuscript ``Software for the analysis and visualization of deep mutational scanning data'' submitted to \textit{BMC Bioinformatics}}
\author{Jesse D. Bloom}

\begin{document}

\maketitle

\subsection*{Reviewer \#1}

Bloom describes a software package for analysis of deep mutational scanning using a novel (for this application) Bayesian/MCMC framework. The general framework appears to be sound and addresses a significant need for these new and highly promising functional assays. The manuscript is well-written and explains the major uses cases of the software in sufficient detail.

Minor comments/suggestions:

1. The notion that differences based on larger absolute counts provide more certainty than those based on small absolute counts is obviously true, but only up to a point. Since deep mutational scanning experiments typically involves exponential amplification of a finite number of selected molecules prior to sequencing, there will be a saturation point (or region) beyond which higher absolute counts no longer provide additional information. This effect might not be negligible in experiments that involve typical cultures of $10^6$ to $10^8$ independently transformed or transduced cells. I would encourage the author to note these effects and perhaps add features or provide guidance to software users for how to estimate or detect the saturation point.

2. On pg 2 � the notion of a mutation being ``beneficial'' is generally taken to be relative to the fitness of the wild-type (non-mutated) gene. Eq (1) is such that $\phi_{r,x} > 1$ does not imply that $x$ is beneficial in this sense. Consider the trivial example of a two site gene with two characters. The input library has one member with a neutral mutation at site 1, one member with a deleterious mutation at site 2 and one WT member. Post-selection $\phi$ for the non-WT character at site 1 will be $\frac{1/1}{1/2} > 1$ even though the non-WT character at site 1 is not beneficial. This should be clarified.

3. Given the substantial increase in runtime and computational complexity of the Bayesian/MCMC approach relative to the simple ratio approach (hours vs fraction of a second), it might be helpful to provide a bit more intuition about what is gained. Is it simply a ``regularization'' where high ratios obtained from low counts are pushed back towards the prior? If so, does the framework really provide more than what could be obtained with a much simpler regularization heuristic?

4. It�s not clear whether \textsl{dms\_inferprefs} outputs the probability of $\phi_{r,x} < 1$ (or $\phi_{r,x} \ne 0$). This is likely to be a useful feature.

5. It�s not clear whether the probability values for $\Delta\pi_{r,x} > 0$ or $< 0$ for \textsl{dms\_inferdiffprefs} should be interpreted as nominal or corrected. Providing false discovery rate estimates might be helpful for interpretation (also applies to 4.).

6. The function of \textsl{dms\_merge} isn�t explained in the paper. Is it simply averaging the two replicate profiles? Could additional statistical power be gained by explicitly modeling replicates within the Bayesian framework?

Level of interest: An article of importance in its field

Quality of written English: Acceptable

Statistical review: Yes, and I have assessed the statistics in my report.

\subsection*{Reviewer \#2}

The author describes a Python package, \textsl{dms\_tools}, which is used to infer character preferences using deep mutational scanning data. This software package allows the user to utilize the method described in Bloom 2014 (\textit{Mol Biol Evol}).

I found the paper easy to read and understand, and the algorithm is explained in a clear and logical manner. Simulations show clearly that this method is an improvement over using enrichment ratios to compute site-specific preferences. I have several questions that I would like the authors to comment on, all of which can be considered discretionary revisions:

Question 1:
On page 2, in the second paragraph, a deep mutational scanning study is described. The average mutation rate is $\mu=1/L$, where $L$ is the length of the gene. It is unclear to me why this is so � intuitively one might think that a longer gene may have more mutations, given that mutation is a random process. More generally, could the author provide more details on the range of data one would expect from typical deep mutational scanning experiments?

Question 2:
The author states that for a depth of $10^6$, we expect about 2000 counts of non-wild type codons for the average gene. The author arrived at this number by multiplying the total number of reads by the probability of a mutation. This is reasonable if each read comes from a different DNA fragment, each with probability $\mu$ of mutating. However, if multiple reads all cover the same DNA fragment, would this relationship still hold? The author should comment on whether this situation is likely to occur in practice and, if so, how it affects the method.

Question 3:
Related to question 1: How does the author think that lower-quality data, either by experimental design limitations or by the variance in counts between sites, affect the inference and downstream interoperation of site-specific preferences? How sensitive is the algorithm to the prior for different qualities/quantities of data?

Question 4:
How robust is the method to the prior assumption that mutagenesis introduces all mutations at equal frequency? The author briefly discusses this issue on page 6, and I think it would be informative to address this question with an additional simulation. Does the present method still outperform the naive ratio method even when data are not generated under the model used in the inference?

Level of interest: An article whose findings are important to those with closely related research interests

Quality of written English: Acceptable

Statistical review: No, the manuscript does not need to be seen by a statistician.


\end{document}  